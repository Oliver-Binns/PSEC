\documentclass{article}
\usepackage{graphicx}
\usepackage[nottoc,numbib]{tocbibind}
\usepackage[a4paper, total={6in, 8in}]{geometry}
\usepackage{subcaption}
\usepackage{titling}
\newcommand{\subtitle}[1]{%
  \posttitle{%
    \par\end{center}
    \begin{center}\large#1\end{center}
    \vskip0.5em}%
}
\newcommand{\myparagraph}[1]{\paragraph{#1}\mbox{}\\}

\title{Topics in Privacy \& Security}
\subtitle{Trust-Enhanced Reputation Metrics}
\author{Y1481702}
\date{\today}
\setlength\parindent{0pt}
 
\begin{document}
\begin{titlepage}
\maketitle
\tableofcontents
\end{titlepage}

%Your report should not exceed four A4 pages (minimum font size 11pt), not counting the screenshots and plots from parts (ii) and (iii) of the question, and not counting the references.

\section{Description} %5 Marks
%Describe how the tool works
%By means of pseudocode accompanied by a suitable description, explaining how your design takes into account scalability issues (so the tool can handle large input files)

\section{Analysis} %15 Marks
%Use the tool with MAX_RATINGS=5 to analyse the effect of using \alpha=1, \alpha=1.5, \alpha=2 and \alpha=5 on the overall product ratings from the input file Q2Ratings.txt.

%Include screenshots of the tool output; and plots of the overall ratings of the products with IDs 4, 7 and 29 for these \alpha values. Discuss the differences between the results obtained for different \alpha values, and between these results and the results obtained for the arithmetic mean.

\section{Simulated Attacks} %10 Marks
%Simulate a self-promoting attack to improve the rating of the product with ID 4 by adding n fake top ratings for this product at the end of Q2Ratings.txt from part (ii) of the question. Each such rating should have a new customer ID, corresponding to a new customer account created by the attacker(s).
%Plot the increase in the overall rating achieved by attacks of size n=5, n=10, n=15, n=20 and n=25 for the four \alpha values from part (ii) of the question and for an overall rating based on the arithmetic mean.

%Repeat the experiments for slandering attacks on product with ID 29, plotting the decrease in the overall rating achieved by slandering attacks of size n=5, n=10, n=15, n=20 and n=25 for the four \alpha values from part (ii) of the question and for an overall rating based on the arithmetic mean.

\section{Results}
%Discuss the results from part (iii) of the question. Explain what value of the parameter \alpha offers the best protection against the attacks you experimented with, and assess the advantages and disadvantages	of different values for \alpha.

\newpage
\raggedright
\bibliography{Report}{}
\bibliographystyle{ieeetran}
\newpage
\section{Appendix}
%You must attach as an appendix to your single-file submission the source code for the tool. If needed, use a ZIP archive to submit all components of the assessment as a single file.

\end{document}